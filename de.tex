\makecvtitle % Print the CV title

%----------------------------------------------------------------------------------------
%	WORK EXPERIENCE SECTION
%----------------------------------------------------------------------------------------
\section{Beruflicher Werdegang}

\cve{Jan 2019 -- Heute}{Softwareentwickler}{Panasonic Europe Ltd.}{Hamburg}
\cvitem{}{
    Ich arbeite an einer dockerisierten, Kotlin-basierten Spring Boot-Webanwendung mit einem Angular-Frontend,
    welche Genehmigungsprozesse über Camunda BPMN mit dynamischen Regeln auf der Basis von DMNs steuert.
    Kürzlich sind wir in die AWS-Cloud migriert und verwenden Terraform, um die Infrastruktur zu definieren.
}
\cvspace

\cve{Apr 2017 -- Dez 2018}{Projektmanager}{Enghouse Networks (Germany) GmbH \footnotesize (ehemals XConnect GmbH)}{Düren}
\cvitem{}{
    Ich leitete ein Projekt mit mehreren großen Telekommunikationsanbietern und setzte ihre Anforderungen
    auf Basis von JIRA um. Intern war ich für die Planung und Entwicklung eines Nachfolgeproduktes für
    automatisierte Auskunftsverfahren nach §112 TKG verantwortlich und setzte dabei, u.a., CQRS, Apache Camel,
    Event Sourcing, Vaadin und OSGi ein.
}
\cvitem{\iconTestimonial}{\link{https://mcarle.io/assets/testimonials/enghouse.pdf}}
\cvspace

\cve{Jan 2012 -- Mär 2017}{Softwareentwickler}{XConnect GmbH \footnotesize (ehemals sms eSolutions GmbH)}{Düren}
\cvitem{\footnotesize \emph{20h/Woche}}{
    Ich war verantwortlich für einige Java EE-Webanwendungen für den B2B Telekommunikationsmarkt in Deutschland.
    Neben der Pflege und Erweiterterung bestehender Anwendungen, plante, dokumentierte und implementierte ich zusammen
    mit meinen Kollegen neue Anwendungen. Darüber hinaus fungierte ich auch als 2nd Level Support per Mail und Telefon.
}
% TODO: mcarle.io or github.com/mcarleio/cv (with tag?) ? 
\cvitem{\iconTestimonial}{\link{https://mcarle.io/assets/testimonials/sms-to-xconnect.pdf} (Zwischenzeugnis)}
\cvspace

\subsection{Start-up}
\cve{Mär 2014 -- Jun 2016}{Mitgründer und CTO}{Bontoo Mobile Development UG (haftungsbeschränkt)}{Würselen}
\cvitem{}{
    Das Start-up konzentrierte sich auf die Entwicklung von mobilen Anwendungen im B2B-Markt.
    Mein Verantwortungsbereich war das IT-Projektmanagement und die Entwicklung von Backend-Systemen und
    REST-Schnittstellen, auf welche die mobilen Anwendungen zugreifen konnten. Mit potenziellen Kunden tauschte ich
    Ideen darüber aus, wie eine mobile Anwendung ihre tägliche Arbeit erleichtern könnte.
}
\cvspace
\newpage

\subsection{Ausbildung}
\cve{Jul 2009 -- Jan 2012}{Mathematisch-technischer Softwareentwickler (MATSE)}{sms eSolutions GmbH}{Düren}
\cvitem{}{
    Ich begann meine Karriere mit einem dualen Studium, d.h. einer Ausbildung zum 
    Mathematisch-technischer Softwareentwickler (MATSE) und parallel dazu einem Bachelor-Studium
    in Scientific Programming an der FH Aachen.
}

%----------------------------------------------------------------------------------------
%	EDUCATION SECTION
%----------------------------------------------------------------------------------------
\section{Bildung}

\cve{2012 -- 2017}{Master of Science}{RWTH Aachen University}{Aachen}
\cvitem{}{Studiengang: \emph{Informatik}}
\cvitem{\iconGraduation}{Masterarbeit: \emph{Unterstützung der Fahrzeugdiagnose in Werkstätten durch Wearable Devices}}
\cvitem{\iconBuilding}{Ausschreibende Firma: \emph{DSA Daten- und Systemtechnik GmbH}}
\cvspace

\cve{2009 -- 2012}{Bachelor of Science}{FH Aachen}{Jülich}
\cvitem{}{Studiengang: \emph{Scientific Programming} \footnotesize (duales Studium)}
\cvitem{\iconGraduation}{Bachelorarbeit: \emph{Entwicklung und Optimierung einer Software zur performanten Speicherung und Verarbeitung von Massendaten für eine offene Systemlandschaft}}
\cvitem{\iconBuilding}{Ausschreibende Firma: \emph{Enghouse Networks (Germany) GmbH}}
\cvspace

\cve{2001 -- 2009}{Abitur}{Archigymnasium Soest}{Soest}
\cvitem{}{Leistungskurse: \emph{Informatik und Mathematik}}

%----------------------------------------------------------------------------------------
%	CERTIFICATES SECTION
%----------------------------------------------------------------------------------------
\section{Weiterbildung}

\cvitem{2021}{\textbf{AWS Certified Solutions Architect - Associate}, \emph{AWS Certification}}
\cvitem{\iconCertificate}{\link{https://mcarle.io/assets/certificates/aws-solution-architect.pdf}}

\cvspace

\cvitem{2012}{\textbf{In-Memory Data Management}, \emph{openHPI, Prof. Dr. h.c. mult. Hasso Plattner}}
\cvitem{\iconCertificate}{\link{https://mcarle.io/assets/certificates/in-memory-data-management.pdf}}


%----------------------------------------------------------------------------------------
%	SKILLS SECTION
%----------------------------------------------------------------------------------------


\section{Technische Skills}

\cvdoubleitem{}{\strut\subsectionstyle{Sprachen}}{\cbstart}{\strut\subsectionstyle{Plattformen}}
\cvdoubleitem{> 10 Jahre}{Java, JavaScript, HTML, Bash, SQL}{> 10 Jahre}{Linux, Windows}
\cvdoubleitem{> 2 Jahre}{Kotlin, TypeScript}{> 2 Jahre}{AWS, Android}
\cvspace

\cvdoubleitem{}{\strut\subsectionstyle{Backend Frameworks}}{}{\strut\subsectionstyle{Datenbanken}}
\cvdoubleitem{> 10 Jahre}{Hibernate}{> 9 Jahre}{Oracle}
\cvdoubleitem{> 6 Jahre}{Java EE}{> 5 Jahre}{MySQL, MariaDB, H2}
\cvdoubleitem{> 2 Jahre}{Spring Boot, Camunda}{> 3 Jahre}{PostgreSQL, Derby}
\cvdoubleitem{> 1 Jahr}{Quarkus, OSGi, NodeJS}{> 1 Jahr}{NoSQL (Neo4j, MongoDB, Redis)}
\cvspace

\cvdoubleitem{}{\strut\subsectionstyle{Tools}}{}{\strut\subsectionstyle{Frontend Frameworks}}
\cvdoubleitem{> 10 Jahre}{JIRA, Confluence, Maven, Jenkins}{> 6 Jahre}{JSF}
\cvdoubleitem{> 5 Jahre}{Git, GitLab, Mercurial}{> 3 Jahre}{Angular, Vaadin}
\cvdoubleitem{> 3 Jahre}{Docker}{}{}
\cvdoubleitem{> 1 Jahr}{Terraform}{\cbend}{}

%----------------------------------------------------------------------------------------
%	SOFT SKILLS SECTION
%----------------------------------------------------------------------------------------

\section{Soft Skills}
\cvitemize{Eigeninitiativ}{Sehr flexibel}{Hohes Engagement}
\cvitemize{Von Natur aus neugierig}{Hoher Teamgeist}{Gutes Konfliktmanagement}
\cvitemize{Kommunikativ}{}{}


%----------------------------------------------------------------------------------------
%	LANGUAGES SECTION
%----------------------------------------------------------------------------------------

\section{Sprachen} 

\cvitem{Deutsch}{Muttersprache}
\cvitem{Englisch}{Fortgeschritten}


%----------------------------------------------------------------------------------------
%	OWN PROJECTS SECTION
%----------------------------------------------------------------------------------------

\section{Eigene Projekte}

\cvitem{\textbf{Cryptonify}}{
    Ein plattformunabhängiger, voll ausgestatteter Offline-Passwortmanager mit Synchronisierung, Browser-Erweiterungen
    und einer Android-App.
}
\cvitem{\iconLink}{\link{https://mcarle.io/project/cryptonify}}
\cvspace

\cvitem{\textbf{GetFav}}{Ein einfach zu benutzender Service, um jedes Favicon von jeder öffentlichen Website abzurufen.}
\cvitem{\iconLink}{\link{https://mcarle.io/project/getfav}}
\cvspace

\cvitem{\textbf{GitHub}}{
    Auf meinem GitHub-Konto veröffentliche ich kleinere Projekte wie \emph{Strix}, eine leichtgewichtige JPA-Transaktionsbibliothek,
    oder \emph{Sciurus}, eine Sammlung nützlicher Aspekte, um, z.B., die Ausführungszeiten von Methoden zu überwachen.
    Dabei ist mir eine hohe Testabdeckung und Dokumentation wichtig.
}
\cvitem{\iconLink}{\link{https://github.com/mcarleio}}

%----------------------------------------------------------------------------------------
%	INTERESTS SECTION
%----------------------------------------------------------------------------------------

\section{Interessen}

\cvitemize{Badminton}{Cloud \& Serverless}{Softwareentwicklung}
\cvitemize{CI/CD}{Java User Groups}{Konferenzen}
\cvitemize{Lesen}{Gerätetauchen}{}

%----------------------------------------------------------------------------------------